%% einleitung.tex


\chapter{Einleitung}
\label{ch:Einleitung}
%% ==============================

Die Eigennamenerkennung (Named Entity Recognition oder NER) ist eine traditionelle Aufgabe der Computerlinguistik mit dem Ziel der Identifikation und Klassifikation von Eigennamen innerhalb eines Textes. Üblicherweise verwendete Klassen sind Personennamen, Ortsnamen und Organisationen, jedoch werden teilweise wesentlich feinere Differenzierungen vorgenommen.

Eine Unteraufgabe der NER ist das Erkennen von verschachtelten Eigennamen (Nested Named Entity Recognition oder NNER), also Eigennamen der Form \emph{Universität Trier}, wobei die innere Entität \emph{Trier} gesondert von der äußeren Entität \emph{Universität Trier} ausgezeichnet werden soll. Obwohl verschachtelte Eigennamen sprach- und domänenabhängig einen erheblichen Anteil der Gesamtheit ausmachen können, wird dieser Aufgabe in der Forschung nicht immer eine gleichermaßen große Bedeutung zugesprochen.

Historisch wurden verschiedene Methoden angewendet, um diese Aufgaben zu lösen. In der modernen Computerlinguistik werden häufig künstliche neuronale Netze eingesetzt, da sie in vielen Anwendungsbereichen zu den besten Ergebnissen führen. Auf dieser Grundlage wurde das Sprachmodell BERT entwickelt \parencite{devlin2019bert}, das unter anderem für (N)NER verwendet wird.

In \textcite{li2019unified} wurde ein einheitliches Framework auf Basis von BERT entwickelt, das für NER und NNER gleichermaßen geeignet sein soll. Da die Ergebnisse vielversprechend sind, aber noch nicht mit deutschen Korpora getestet wurden, wird dies im Rahmen dieser Arbeit nachgeholt. Das Ziel ist, eine erste Einschätzung zu erhalten, ob sich die Ergebnisse von \Citeauthor{li2019unified} auch auf deutsche Korpora übertragen lassen.

% %%% End:
