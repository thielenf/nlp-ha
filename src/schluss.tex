%% schluss.tex


\chapter{Fazit}
\label{ch:Fazit}
%% ==============================

In dieser Arbeit wurde das von \textcite{li2019unified} entwickelte einheitliche Framework für NER und NNER erstmalig auf deutschen Korpora angewendet und evaluiert. Obwohl \Citeauthor{li2019unified} sämtlichen Quellcode offenlegen, war die Anwendung mit großem Aufwand verbunden, was einerseits auf die rasche Weiterentwicklung und begrenzte Abwärtskompatibiltät im Bereich des Machine Learning mit Python zurückzuführen ist. Andererseits waren entscheidende Details der erforderlichen Datenformate bedauerlicherweise nicht nachvollziehbar in \textcite{li2019unified} oder dem dazugehörigen Quellcode dokumentiert. Die anschließende Evaluation hat gezeigt, dass die Performance auf den deutschen Korpora vergleichbar mit den von \citeauthor{li2019unified} erhaltenen Ergebnissen ist. Gleichzeitig konnten aufgrund der Komplexität einer solchen Untersuchung nicht alle Facetten beleuchtet werden, die für eine allgemeingültige Aussage über die Performance des Frameworks, sowie den potenziellen Mehrwert im Bereich der (N)NER notwendig wären.

%%% End: 